\documentclass[conference, compsoc, twoside]{IEEEtran}
\usepackage{graphicx}
\usepackage{array,etoolbox}
\usepackage[labelsep=period]{caption}

\hyphenation{op-tical net-works semi-conduc-tor}
\begin{document}
\preto\tabular{\setcounter{magicrownumbers}{0}}
\newcounter{magicrownumbers}
\newcommand\rownumber{\stepcounter{magicrownumbers}\arabic{magicrownumbers}}
\title{\Huge Intro to Empirical Software Engineering, What We Know We Don't Know : A Review}
\author{
\IEEEauthorblockN{Muh. Riansyah}
\IEEEauthorblockA{
	Faculty of Computer Science - University of Indonesia\\
	Kampus UI, Depok 16424\\
	muh.riansyah@ui.ac.id
}}

\maketitle
\providecommand{\keywords}[1]{\textbf{\textit{Index terms---}} #1}
\begin{abstract}
Almost everything in Software Engineering is belief.
\end{abstract}

\keywords{Belief, Empirical}
\IEEEpeerreviewmaketitle

\section{Introduction}
I actually want to start us off with an exercise. 
Imagine you're looking at a function and it's 40 lines long.
A pretty long function and you can break it down into say 14 line functions.
Taking a big function and making some smaller functions. 
So exercise one, raise your hand if you think that the small functions will be easier to work with in the big functions. 
I see most of you raise your hands cool down yeah on average okay so
question number two it's flu season raise your hand if you think that a
vaccine prevents diseases thank God I think that's everybody don't
have to kick anyone out so last question
raise your hand if you think it is more
likely if you believe more strongly that
small functions are easier then you
believe that vaccines prevent diseases
sorry oh is that raise your hand if you
are more confident in your belief that
small functions are good then that vaccines prevent diseases and I see a
couple of very brave souls have raised
their hands but almost everybody else
who raised their hands for both times
put it down why is that why can we
believe two things competent two things
but believe one more well for vaccines
we have we've got medical studies we
have historical data we have the
elimination of smallpox we have clinical
trials we just have so much but for small versus big functions we have a feeling in experience our opinions and we might have some logic like it's obvious that small functions are easier, but then again it also sounds obvious that injecting a virus into your veins isn't going to make you healthier right our logic can often be flawed now.
This talk is not about small versus big functions not telling you to write big functions this talk is not called write big
functions but I do want to underscore some really important point almost everything in software is a
belief it is something we have
experience about it is something we have
opinions on but it's not something we
have hard data in most cases we just
don't know but we can find out we find
out through empirical engineering or EMSE
for short that is the science of taking
claims about software and dissecting
them testing them observing them to find
out what's really true and what just
feels good my name is Hillel Wayne and I
am here to talk to you about what this
is how we do it and it's why it's so
important and some of the things we've
learned on the way but first some
bookkeeping I'm going to be
name-dropping about 35 different studies in this talk you don't have to write them down if you go to this link I have
every single one online annotated and linked you can just go there and find
them and look at them that way also every question being asked through the app.
I will be answering if I don't get to it in this time I will be also
uploading all the answers to that linked
to so you'll be able to see every
question people ask and the researched
answers there and with that we're ready
to begin so I think the first question
you have to ask is why 
\section{Reason to do empirical research} % (fold)
why do we care about this why is it so important to be empirical and I see three reasons
\begin{enumerate}
  \item Inefficient affect on GDP.
The first one is the easy one I'm a
developer I want to get better I want to
know what works and what I just feels
good by studying empirically.
I can find that out but that's probably not that convincing to any of you right.
I mean it's one thing to sort of have a more realistic argument of oh we should be better but quite different to actually do something the other reason extra density that's
important is financial the tech industry makes up about 10\% of the United States GDP that's 1.5 trillion dollars a year
if we are 1\% and that's really low estimate 1\% inefficient that's the GDP of Iceland we are setting on fire every single year but that's it the large-scale that's not local to us and here's what I think is.
  \item To protect ourselves.
The most important the most subtle but most valuable reason to study the empiricism of software we do it to protect
ourselves you see the most common most popular paradigm in software is charisma-driven-development.
There are experts who are good at speaking we're good at writing and they tell us what we should be doing maybe we do scrum because it works for our company maybe we do it
because that's when everyone else is doing. 
Maybe we need a big data cluster to crunch our gigabytes of data.
Maybe we're just following Google's lead.
Maybe we do object oriented.
Because people stand on commerce stages and say you must use S.O.L.I.D, but yes see empirical engineering just cuts all of
that away it's what helps us distinguish
what's fact from what just a Salesman is
telling us empirical engineering is what
tells us that well in 2014 MacBook Pro
can crunch 50 gigabytes of data 100
times faster than a hundred server spar
cluster it tells us that experts separated by hundreds of miles writing the same kinds of problem will make the same mistakes in the same places with I'm sorry is that like part is that is that something we can like sort of hide ok ok yeah so fundamentally ok yeah I'll just ignore for now no so fundamentally the reason we study ESC is to protect ourselves from the thought leaders and the Predators.
\end{enumerate}

% Reason to do empirical research subsection_name (end)
\section{EMSE is study about People and people is hard to learn}
But doing so is very very hard ! 
it's complicated to study! people are complicated! 
we're studying people and people are more complex than atoms.
I mean, take that original question I asked, our small functions easier than big functions.
Well, how do we even define better ? do we define it with a code metric, like cyclomatic complexity? 
some people do and some people do studies that find that's the case. 
But that just pushing the question back, 
how do you know that cyclomatic complexity is better ? 
we don't. Instead we can find maybe a more goal-oriented result. 

\section{Effects of Clean Code on Understandability}
We say better means well the code is 
\begin{enumerate}
	\item easier to read or
	\item easier to modify or
	\item easier to debug

\end{enumerate}
that (indicator) makes sense to everyone.

Who and turns out that we have not that many studies on whether clean code is actually easier to read and I have looked. i've looked pretty dang hard.
about this and i was able to find a
couple of studies that actually studied
this and they both said the same thing
small functions are easier to read
easier to modify and much much harder to
debug so the evidence is mixed then
again those are small sample sizes with
lots of caveats and well look people are
complicated so complicated we're doing
such complicated things that some
engineers think this is impossible we
cannot study ourselves we cannot get
hard data on what we do and in doing so we've made a mistake that a lot of engineers make a lot of smart people too.
Because we don't know how to do something it can't be done but just as people are very smart they're also very
clever and there are people who just as we put our lives into building
complicated systems have put their lives into studying us.
\section{Type of Research} % (fold)
\label{sec:type_of_research}

% section type_of_research (end)

So I want to give an example of what this looks like there are many different kinds of research we do.
I'm going to break them down to three kinds quantitative, qualitative and code mining.
In terms of what recently looked like this is one of my favorite papers.
ever I know it looked small but it's.
\subsection{Fixing Faults in C and Java Source Code: Abbreviated vs. Full-Word Identifier Names}
well it's four-page forums pages to a sheet double-sided so it's actually about 40 pages in total this amazing paper is called fixing faults in C and Java source code
abbreviated vers forward identifier
names yeah not the most exciting title
and it's a pretty small topic - what
makes it interesting it's interesting
because it shows perfectly how we do
research and why it matters.
let's take a simple example I have a I
have a codebase and one of the variable
names is employer\_number okay
descriptive name descriptive title great
is it easier to debug this verse debug
this emp\_num easier to read the code
easier to find faults what not so this
is what they did they want to study
which of these would be easier to work
with and what they were doing is called a qualitative a quantitative study it's
what we most commonly think of as science we have two groups a control
group and a modified group control group
we do not touch my group we make some
tweak - then we have them both to a task
and see which one does it better in this
case the control group was debugging
code and the other group was debugging
code where all the full names were
replaced with abbreviations then we can
see which one does it better and then we
can know which one's better now I see
some of you looking skeptical. I need me
right? there's lots of what we call
confounding factors things that can
explain our results without our
fundamental premise being measured for
example, experience.
 maybe one group did a better job because they all have 10
years more experience there might also
be alternate explanations
maybe this only matters when you're
working with assembly and if you're
working in Python the difference changes
there are a lot of different confounding
variables that can ruin an experiment if
we aren't careful and if we know them we
can design our experiment to control for
them make sure they don't matter
make sure they don't affect the results
but you have to think of them in advance
so they did they tried to figure out
what could possibly give alternate
explanations for whatever they were
seeing and I'd actually can make this a
quick demo quick exercise.
take 30 seconds think down write down now.
No
maybe ask the person next to you really
want to try to come up with possible
confounding factors things that might
explain the results besides just that
give you all 30 seconds everybody back up eyes back on me great?
everybody hub has everybody to have some
things.
How many people got like one thing. 
How many got like two to four? 
How many got five? six ?
Let me read all the things that they found.
I even compared it to your list
\begin{enumerate}
\item experience level of the developer's
\item education level of the developers
\item programming language used 
\item size of the codebase studied
\item density of bugs in the code base 
\item formatting of the code
\item classification of bug 
\item time of day
\item fatigue level 
\item prom domain (prom?)
\item sample size
\item experience with the problem 
\item social media exposure
\end{enumerate}
% Kejelasan nama variable diuji dengan programmer positif rate on debugging.
How many of those did you miss ?
Yeah for the record, they didn't find any difference.
So it turns out that while we know that descriptive names are really important.
There's no evidence that a full word is required when an abbreviation can fit in and this was only over a hundred people. So it's not a hundred percent validated, but it does show we can study something and get real results about what matters. who here find that result surprising well so today, (hasilnya), It doesn’t really make intuitive sense that a code base that emp\_num is going to be no harder than employee\_number, right ? See ? Quantitative studies are science but they’re not enough.
We also need qualitative studies, this is the studies of people’s experiences
\begin{enumerate}
\item their opinions
\item their ideas
\item how they flow
\end{enumerate}

it is the exploration, (it's) part of science, it’s how we get. 
the ideas we want to test in the first place.
We need to explore, so they explored. 
They did what’s called an ethnography.
they sat down and watched developers debug code in the real world, with 
\begin{enumerate}
\item no sort of controls
\item no suggestions 
\end{enumerate}
just watch them and they saw that the two groups had different ways of debugging.
when you have the full word identifiers
people tended to skim !(baca cepat, hanya melihat hal hal penting) 

They used the name as an anchoring point lexicography or in themselves around the code and
quickly jump between where they thought the bugs might be this worked as a debugging technique.
 
The people who had abbreviations though they more methodically went top-down understanding the context and the full flow of the code. This also worked. So changing the names did subtly change how people debug code but in both cases there was advantages and disadvantages they both worth. 
Qualitative studies let us actually know.
What is we're seeing what is interesting what we want to explore
now both of these studies these clinical
trials these of sonography x' are about
people and people are tricky to study
something that's easier to study that
was code right it just sits there it's
not going to change on us it's not going
to be tired or sleep-deprived and code
also has one bigger one big advantage it
scales if I asked you average code base
what's the unit testing coverage how
would you find that out anybody want to
make raise your hand anyone yep yeah but
where do you get the UM code so the
question becomes how do we I'll just go
ahead.
I think you have the right down the track here so I mean we can try code
that analyzes like our code base to find
the unit testing coverage but how do you
get enough samples how do we know the
average well there's 100 million
repositories on github we've just found
them all crunch the data done problem solved and this does work.

\subsection{On the diffuseness and the impact on maintainability of code smells: a large scale empirical investigation}
Here's another study that was done similar to this on code
smells these were people who looked at
30 open-source projects each of which
had been around for over a decade and
studied what the anti-patterns were what
they did wrong and where the defects in
the code were based on what was changed
their results were twofold one was that
yes if there are code smells the code is
more likely to be buggy in that area
that's probably obvious to a lot of us
the second thing that they found and
this is a little bit more surprising is
that
fixing the anti-patterns did nothing to
the bugs it turns out the two were
correlated certain kinds of code led to
more poorly designed low quality code
and more bugs but they were independent
fixing one didn't affect the other so
this means that we can use code smells
to isolate where we should be looking
for bugs but we can't fix the bugs we
just fixing the code smells we have to
actually figure out what the bug itself
is this is an example of how we can use
code mining to very quickly and
efficiently get insights into how things
work but code mining does this very
effective has traps of its own you see
it's not a controlled environment we're
looking at the field we're looking at
the world and that makes for very
complicated noisy data and we have to be
very careful about that.

\subsection{A Large Scale Study of Programming Languages and Code Quality in Github}
Who's heard of this paper came out in 2017 a large-scale study of programming languages in code quality and github see
a few of you because it was a pretty
moment to study it was the first it only
has about a hundred million lines of
github and it was the first site that
showed clear sophistical significance
between different kinds of programming
languages using commits and bug fixes
and commits they found that functionally
programming languages were safer than
imperative languages static typed
languages had fewer bugs and dynamic
type languages manual memory languages
were buggier then controlled garbage
collected languages this came out a
couple years ago and while the aspectus
was small it was hailed as one of the
best evidences for the port importance
of programming language now there's one
more part of this entire process that I
have neglected to tell you we don't
trust papers a paper is interesting it's
insightful but it's not trustworthy
people researchers make mistakes too in
order to see a paper and actually
benefit from it we have to do it's
called the replication we have to get
another group to do the same experiment
and see if they get the same results
that makes us more confident the study
on naming was done replicated several
times successfully as far as we can tell
that's a pretty consistent
this though just this year we tried to
replicate this a group tried to analyze
the same repositories and get the same
results and in doing so they found a
small problem you see this
imagine you have a commit that looks
like this add in fix operator the other
group was flagging it as a bug because
it had the word fix in it and all about
one third of the commits they studied
were false positives once this was a con
for every school difference went away
they could not find any evidence that
one language was better than any other
language doesn't mean it's not true it
just means you don't have evidence yet
so code mining can be very effective and
get us really deep insights but we have
to be careful we have to make sure that
what we're doing actually makes sense
now something interesting about all of
this that I've just been sharing you
might be noticing a pattern I've named
several things that we think are matter
programming language how we name our
things how we look at code smells all
that stuff and nothing seemed to have a
really strong effect write abbreviations
totally fine code smells don't actually
identify bugs it turns out that this
talk is what we know we don't know and
we know we don't know pretty much anything. 
See ? Software Engineering is a very-very young field. 
Some of the founders are still alive today. 
It's a feel about systems and any system is going to be complicated. 
any system is going to have some things
that are obvious and false and some
things that are insane and totally
correct we don't know that doesn't stop
us from programming we can still build
pretty incredible stuff just as humans
are complicated and clever we're really
good at doing things in on certain
situations so we don't know the answers
and that's okay but just as we don't
know the answers nobody else does either
and that's what it comes back to so many
people tell us here is how you must code
they're the people who tell us you must
use agile but they don't know they're
just saying that they just do believe
that though the people say agile is a
waste of time they don't know that they
just believe it they're just saying that
and that's the key here we don't
actually know anything nobody does and
that means anybody who's certain about
what isn't is it true about software is
probably wrong and probably trying to
sell you something we have to be
methodical we have to understand the
limits of our fields and learn how to
push them we have to be careful and
methodical and explore and short we need
to understand now I've shared some
things that don't work most things don't
work but there are some things that
we've studied that we are pretty sure
make a difference we have done many
experiments in many contexts and they've
all found significant persistent
positive or negative effects I'd like to
share some of you with this to show you
that there is some hope here.

\subsection{Beyond Lines of Code: Do We Need More Complexity Metrics?}
I'm going to also focus on the field that matters
most to me I do what's called formal verification the study of making
programs probably correct unfortunately
that's not been cited by anyone so
complete waste of time but one thing
that has been pretty heavily studied is
defect finding software defects how do
we know where the bugs are in our code
and a second how do you how do we
prevent bugs in the first place
that's what I'd like to talk about these two categories and what we've learned
about them so first question how do we
find bugs I've already shared one thing
that works identifying code smells and
seeing where they are helps us trace down where the bugs are. but that's probably not enough for most people we want an automated tool that helps us more carefully more accurately identify code. that led to an explosion of code measuring techniques.
who here has heard of cyclomatic complexity.
Who's here has heard of function points clean code most
people these are techniques people try
to use to measure the quality of
software and maybe they work but in
terms of finding where bugs are most
likely to be in code there was one
technique that works much better than all of them. Lines of code, more lines, more bugs, now you might feel cheated by this because again we want an automated tool, that we can point our code and find where the bugs are lines of code doesn't help us. Just saying there's a thousand lines probably bug somewhere in there.
Just doesn't really do anything for us and as far as we can tell.
There just really isn't a way to just look at a code base and find where we can find the bugs so we don't look at the code base instead we mind the org chart. 
\subsection{The Influence of Organizational Structure On Software Quality}
you might have heard of Conway's law code reflects
the organization that produced it and it
turns out that is empirically true in
both positive ways and negative ways if
you have code if you have a system in
the organization a functioning
organization that is cross-cutting and
complicated the code for that system is
going to probably be buggy this has been empirically verified if you have a lot of different people that touch a code
base it is more likely to be buggy if
you have a lot of different groups that
touch a code base it is more likely to
be buggy not in the rate of change but
in the rate of types of change and this
is a pretty consistent persistent effect
so it's not necessarily a technical
thing that we look at but the social
thing our hierarchy is our VCS our git
blame that help us identify where the
bugs are going to be that's though in
the general case in the specific case we
know that in certain contexts it's
easier to find bugs for example in a
distributed system about nine out of
every ten critical bugs that crash the
entire distribute system are either
uncaught exceptions you know the kind
you find with the unit test or
configuration errors so if you look at
those two things you'll cut out maybe
90\% of your crashes we also know from
some surveys that about half of the
worst bugs that take the longest to fix
our requirement or design issues so if
you just sit down and dry our decision
table before you start coding you will
probably save your company a few hundred
thousand dollars
but that's all in the finding of bugs
ideally we don't want bugs in the first
place right that's harder there's a lot
of things we've studied on this and most
of them seem like they work and seem to
work in practice for us but when we put
them to the test
they just fall apart take oh no
test-driven development now I'm gonna be
very clear here testing is great
everybody thinks testing is great in
fact it's so great that it's almost
impossible to find Studies on it it's
what's called a parachute study
something so obvious nobody bothers to
study it this term comes from medicine
well there's no double-blind studies
showing that parachute save lives so how
do we know for the record 
I did spend about three days hunting down really old
studies and they all agree that yes testing has an overwhelming benefit keep
writing your tests the question though
is does test and development work better
who here knows the test room development
is aa great most of you for the for the
people don't it's a very tight cycle
where you first write a failing test
then write the code that passes the test
then refactor it's really widely lauded
a lot of people really love it I
personally love it I do it all the time
\subsection{Realizing quality improvement through test driven development}
I recommend friends do it but does it
actually make a difference well we have
one said he's saying yes this came out
in 2006 it was the first long-term study
on test row development it found that it
did reduce defects but also added about
20\% more testing time to your system
which made the effects kind of uncertain
maybe it was the Chi GD maybe was just
we spent more time testing we've done a
lot of follow-up studies since then and
as far as we can tell no there's really
not a difference doesn't make a
difference that much to quality either
as far as we can tell test-driven
development is no better or worse than
any other disciplined controlled testing
technique this is personally a huge
bummer to me because I as I said loved
doing it and it's kind of frustrating to
know that this thing that I know helps
me probably doesn't work but that's
being what empirical means it means
accepting the results accepting the data
you
if we don't see that even if we don't
like the data and it turns out pretty
much every other technique we've studied
pair programming type systems etc don't
really have that much effect either they
feel like they help they probably don't
except for one technical practice there
is one technical practice that we've
studied again and again and know for
certain not just finds and removes bugs
but is dramatically effective at doing
so code review now there are some caveats
here you can't review that much at a
time you can review that many lines of
code at a time but in those constraints
the effect is absolutely enormous most
of the rigorous studies they've seen on
this say it finds about 60 to 80 percent
of all the bugs in the code and even
better than that that's the secondary
effect it turns out that only one out of
every four comments that basically block
the code significant software defects is
about functionality the other three are
about code quality so very roughly for
every bug it finds which is again about
sixty to eighty percent of all of the
bugs it spines about three situations
where we can just make the code better
more maintainable spread knowledge share
knowledge code review is simply
fantastic and no other technical
practice comes close not pairing not TDD
not testing in general not types not
even like formal proofs to be honest
these are still great things and I still
recommend doing them but far and away
TDD is the one technical practice we are
absolutely 100\% certain is effective
nothing else comes close to code review
at the end I said though technical
practice for a reason we haven't studied
software engineers as much as we really
should have as I've made clear but we
have to be knowledge workers in general
we've studied them for a hundred years
and we know without a doubt that there
are three things that have a profound
impact on the output of any possible knowledge worker any possible manual
laborer anyone doing anything 

\subsection{Impact of a Night of Sleep Deprivation on Novice Developers Performance}
Sleep deprivation stress levels and hours worked and these effects are absolutely
enormous an unstressed well-rested oh
not overworked team that is happiness
job will produce orders of magnitude
better code better output better systems
than otherwise as just one example of
one of the few cases we've studied
software engineers in this context this
was a study about what happens if you
skip a night of sleep one night so
you know your hackathon what you're
doing at the end of the hackathon
instead of at the beginning of the
hackathon and if you skip one hour of
sleep if you sleep one night of sleep
for the first hour of coding after that
just the first hour on simple tasks
you're about half as productive also
other studies show that if you miss
about a week of two hours of sleep at
night you are basically about as bad off
as a person who skipped an entire night
of sleep so chronic sleep deprivation
can be just as bad also also it turns
out that when you are sleep-deprived you
can't tell your work is worse so if your
team is sleep-deprived there is
literally nothing you can do to make up
for that no practice will make your code
any better then they would make if they
were well rested and well it's not just
sleep I mentioned also time worked and
hours worked right and stress one of my
favorite studies that's come out
recently is the gameís Sutra study on
game developers they interviewed 700
game developers on 700 separate teams
and we know that down to 270 different
code bases total seven different games
and among other things they found that
when a team entered crunch mode that is
over work to get a game done in time
they produced games that were worse on
every single metric reviewer scores
profitability user satisfaction sales
everything then the teams that simply
cut scope or push their deadlines those
groups were burning more time and money
not to mention the health and safety of
the developers on a worse outcome so the
question you've probably heard that
correctness you should be doing testing
or review or pairing why haven't you
heard about sleep why haven't you heard
about stress levels for correctness a
lot of reasons to be honest because
these are a long-term subtle effects as
opposed to short-term ones because they
are diffuse and insidious because it's
very hard to trace them back to their
source and because it's not in our
control things like stress and sleep are
a product of things like
deadlines scope creep bad managers bad
company culture things that our
organization level social not technical
you see there are some things that US
engineers can do that will improve our
code like code review but ultimately at
its core software engineering is
knowledge work
it's about us putting our minds to the
best use we can and I find that beautiful it really exalts what exists
that humans can do but at the same time
it means that anything that impairs our
ability to think is going to cause much
worse effects than anything else can so
yes if we want higher quality we need to
do our code review we need to be careful
but if you really want high quality and
high productivity well that can't be
demanded of engineers it has to be
enforced at the organization the change
must come from the top now that is just
software engineering just empirical
engineering in software defects there's
other fields we've studied - education
human-computer interfaces performance
all that stuff.
I shared software defect because that's
what matters most to me I don't know
what matters most to you maybe it's
something else all I can do is encourage
you to look for yourself
I encourage you that this is even worth
looking at in the first place hopefully
I've done that a little at least a
little bit of that hopefully I've
convinced you of the value here if so
I'd like to end by talking about where
we can get started what's the best
introduction to both doing the research
and finding it so there's two books I
definitely strongly recommend the first
is making software this is how I got
into it the first half is about the
practice of research how we do the
research the pitfalls everything the
second half is the things we've learned
this book is absolutely fantastic I
reread it once a year if any of you have
a safari subscription it's free online
there too they also have a site never
work in theory org which has high
quality open source research I'd also
recommend reading that other book is a
counterpoint the leprechauns of software
engineering this is by person laurent
bosavi who is skeptical the idea of
empirical engineer
obviously I disagree with him on that
but what he does in this book is show
how it is that people misinterpret
research how claims turn into urban
legends so it is a very good book for
learning the methodology of evaluating
research that's mostly about how we do
research in terms of finding it
that's a trickier problem who here has
heard of the academia industrial complex
basically goes like this
almost all research is done by
universities universities have their
stuff published in scientific journals
to read a scientific journal you have to
pay either\$30 per article or belong to
an organization that pays\$10,000 a year
for access I'm guessing you most of you
aren't in that so you can find the paper
you can read the abstract you can't
actually read the paper there are a few
ways around this though if you have an
ACM digital membership that's about a
hundred dollars a year you can read all
the memberships in their system if you
go to this place called the archive a
lot of scientists in protest and
rebellion of the system upload their
preprints there if you go to the
scientists actual website they probably
have their stuff hosted there too if you
email the scientists they'll happily
share it but by far the most efficient
most effective and easiest way is to use
SCI hub SC i - h ub if you put in a
paper into the site it will just
immediately give you the entire article
no problems no questions asked the
problem is I can't actually recommend
this because it's incredibly fast and
convenient and workable and has great UI
but it's also a little bit illegal
because you know copyright rulings so
you really you're really supposed to you
know if you want to be really moral
about this you kind of have to pay the
\$30 per paper so I'm definitely don't go
to that website definitely but it all
you can and don't go to there and don't
like follow them on Twitter just don't
so in 

\section{Conclusion}
Software Engineering very powerful very difficult.
But it helps us distinguish what is correct from what we believe and what is useful from.
What is either negative or uncertain it's great for I guess humility and actually improving and protecting ourselves now a couple of things to wrap up really quickly.
First as mentioned you can go to the site and you can see all the reference.
You can also read the references for yourself and it would recommend that everything I've shared
has been colored by my opinions, how I see the world my own biases maybe when
you read it you'll come up with
something different maybe you'll think
it's correct maybe you'll think it's garbage.
I do recommend though checking for yourself because you should see for yourself what the research says and not
just trust a person on the stage selling their consult shilling their consulting business speaking of shilling I'd like
to end by clearly talking about what I
do I work in a field called formal
methods that is sort of the art and
science of producing large-scale
bug-free designs essentially software blueprints.
I teach workshops and consult for companies
clients have included Netflix Cigna
protocol labs Skala tea medium math so
far they seem pretty satisfied with me
so probably a good sign if you're
interested in this just either go to the
site or come talk to me after I'll
happily answer any questions on what I
do and how it works and of course you're
always welcome for the rest of the
conference to ask me any questions you
have about empirical engineering and
with that my name is Helene and thank
you for listening to my talk

\bibliographystyle{IEEEtran}
\end{document}
